\section{Samhälleliga och etiska aspekter}

Vanadin lades i januari 2018 i EU:s lista över kritiska material under direktivet om en cirkulär ekonomi \cite{Navigation2018}. 
Som en följd av detta har företag som exempelvis Scandivanadium och EU Energy Corporation via Bergsstaten ansökt om undersökningstillstånd att utvinna just vanadin ur svenska mineralfyndigheter på Österlen i Skåne, respektive utkanten av Östersund i Jämtland. Dessa beslut möts ej sällan av kontrovers där missnöje yttras genom protester från bofasta individer i områdena. \cite{NohrstedtLinda2018}. 
För att tackla problemet med att försöka uppnå en cirkulär ekonomi kan det därför finnas ett intresse i att utvinna vanadin på annan väg, nämligen deponerat LD-slagg istället för att bryta helt ny mineral.

Det finns en  väldigt stor industri som tillämpar LD-processen för att tillverkar stål. I Sverige tillverkades 2017 4,713 miljoner ton stål, dock inte exklusivt via LD-processen \cite{WorldSteelAssociation2015}.Som tidigare nämndes i bakgrunden producerar stålproduktionen mycket LD-slagg som läggs på deponi. En felaktigt hanterad deponi skulle kunna leda till läckage av tungmeteller eller fosfor till närliggande vattendrag eller jordmån. Det skulle även kunna vara så att deponin läggs på ett ställe där den blir skrymmande eller stör utsikten. 

Genom att använda LD-slagg som syrebärare i förbränningsprocesser och vidare laka den brända slaggen för utvinning av vanadin ger en hållbar och cyklisk process som utnyttjar restprodukter från en annan process som är lättillgänglig och finns i stora kvantiteter. Förutom att sluta materialkretslopp avseende vanadin så kan även fosfor med fördel lakas vid det här stadiet. Det medför även ett minskat behov av att bryta malm med avsikt att utvinna vanadin eller fosfor. 