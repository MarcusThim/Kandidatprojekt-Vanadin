\section{Samhälleliga och etiska aspekter}
Det finns en  väldigt stor industri som tillämpar LD-processen för att tillverkar stål. Till följd av detta finns även en väldigt stor sidoström innehållande stålslagg, kallat LD-slagg. Detta slagg läggs i nuläget på deponi, vilket skulle kunna vara en negativ samhällelig aspekt om det hanteras illa. En felaktigt hanterad deponi skulle kunna leda till läckage av tungmeteller eller fosfor till närliggande vattendrag. Det skulle även kunna vara så att deponin läggs på ett ställe där den blir skrymmande eller stör utsikten. 

% En mer etisk aspekt av deponin vore i fallet att exempelvis tungmetaller och fosfor läcker ut i närliggande vattendrag eller jordmån.

Att använda slaggen som syrebärare är ett effektivt sätt att utnyttja restprodukter från en annan process, som är lättillgänglig och finns i stora kvantiteter. Går det dessutom att laka vanadin och kanske även fosfor bidrar det till en mer hållbar och cyklisk process. Det medför även ett minskat behov av att bryta malm med avsikt att utvinna fosfor. 





% Genom att använda LD-slagg som syrebärare i förbränningsprocesser och vidare laka den brända slaggen för utvinning av vanadin ger en hållbar och cyklisk process som utnyttjar restprodukter från en annan process som är lättillgänglig och finns i stora kvantiteter.