\section{Problem}
%Den huvudsakliga uppgiften består i att jämföra möjligheterna att utvinna vanadin ur LD-slag som har genomgått olika behandlingar innan lakning.

%Deluppgift ett kommer bestå i att utföra en rad experiment som undersöker under vilka förhållanden och i vilket medium, d.v.s syra, som lakningen är mest gynnsam.

     

%Deluppgift ett kommer att bestå av att utföra experiment på vilken koncentration av svavelsyransyran som är mest gynnsam för lakning. 

%Deluppgift två kommer att vara att ta reda på vilken temperatur som är mest gynnsam för lakningen. 

%ändra detta då vi bara kommer titta på syra i nuläget

%Deluppgift tre kommer innefattar att identifiera vilken förhållande mellan syra och LD-slagget som är den mest optimala. 

%Deluppgift fyra innefattar att analysera de olika filtratet från lakningen med hjälp av AAS. Samt att analysera, förutsatt att det inte är helt upplöst av syran, lakgodset med hjälp av XRD. % Hade velat flytta "lakgodset" så att det står mellan analysera och ,

%I mån av tid kan det även testas huruvida det är möjligt och i vilken mängd fosfor kan urlakas från LD-slagget. 
%tid kan det även testas huruvida det är möjligt och i vilken mängd fosfor kan urlakas från LD-slagget. 

%Utöver de test som kommer göras för att analysera mängden utvunnen vanadin kommer det även testas vilken halt fosfor som erhålls i samma prov.


Uppgiften kommer vara att utforska möjligheten att urlaka vanadin ur LD-slagg som har använts som syrebärare i en fluidiserad bäddpanna. Där experiment utförs och olika parametrar används i experimenten. Det som kommer testas är vilken koncentration av svavelsyra, vilken temperatur samt vilket förhållande mellan flytande och fast fas som kommer att ge optimalt resultat för utvinningen. I mån av tid kommer det även testas om möjligheten finns att laka fosfor ur LD-slaggen.
