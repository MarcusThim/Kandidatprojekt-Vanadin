\section{Bakgrund}
%Stefan: https://www.accessscience.com/content/steel-manufacture/653800


%%Vid framställning av stål bildas diverse restprodukter som exempelvis gasreningsstoft, gasreningsslam och just slagg. Gasreningsstoft är de små partiklar som följer med de gaser som bildas i ståltillverkningens varma processer. Dessa avskiljs i hög utsträckning tillsammans med rökgaser i diverse filtreringsanordningar. Stoften kategoriseras som torra eller våta och efter rening  bildar de gasreningsstoft respektive gasreningsslam. [Jernkontoret - Gasreningsstoft och -slam blir nya råvaror][Jernkontoret handbok] 

%Stålslagg förekommer i en rad olika typer som t.ex. ljusbågsugnsslagg och LD-slagg\cite{Pehlke2014a}.
%Slaggen kategoriseras utifrån vilken typ av ugn som använts vid framställning av råjärnet. Ljusbågsugnsslagg är restprodukt av skrotbaserad ståltillverkning medan LD-slagg är restprodukt i malmbaserad ståltillverkning\cite{Pehlke2014a}. Slagg kan brukas i diverse tillämpningar som exempelvis konstruktionsmaterial eller asfalt; detta för att minska uttaget av jungfruliga resurser. Dessutom använder Stålverken begagnat material som skrot för tillverkning av nya produkter; detta för att nyttja resurserna till sitt yttersta. [https://www.jernkontoret.se/sv/stalindustrin/tillverkning-anvandning-atervinning/restprodukter/slagg/][https://www.jernkontoret.se/sv/stalindustrin/tillverkning-anvandning-atervinning/atervinning-av-jarn-och-stal/] 

%Svenskt ståls branschorganisation Jernkontoret hävdar trots detta att 20\% av avfallet, där spår av vanadin kan hittas, hamnar på deponi hos antingen kommunal deponeringsplats eller en extern aktör\cite{PontusWestrin}. Det tål också att nämnas att det finns tillgängligt flera ton av gammalt deponerat slagg till förfogande som syrebärare följt av lakmaterial. %Notera att man här inte vet vad varken en syrebärare eller ett lakmaterial är. Kanske ta bort meningen alternativt flytta ner den?

%[https://www.jernkontoret.se/sv/stalindustrin/tillverkning-anvandning-atervinning/restprodukter/]
%Båda dessa citerade 04-02-2019 

%\subsection{Som syrebärare i förbränningsprocesser}
%Det har visat sig att LD-slagg har potential som en billig syrebärare i kemcyklisk förbränning (CLC), där två stycken fluidiserande bäddreaktorer nyttjas för att separera CO$_2$ vid förbränning \cite{Xu2017}. Syrebäraren, som i detta fallet är LD-slagg, oxideras i ena reaktorn genom tillförsel av syre. Denna förflyttas sedan till den andra reaktorn där bränslet reducerar LD-slagget, för tillbakaförsel till första reaktorn.  %Har freestylat detta. Glöm inte att jämföra med källa för att se så det stämmer!
 
%Syrebärare kan även nyttjas i Oxygen Carrier Aided Combustion (OCAC), d.v.s. förbränning i en fluidiserad bädd där bäddmaterialet består av just en syrebärande metall.
%Syftet är att transportera syre till bränslet i understökiometriska områden för att på så vis öka pannans effektivitet\cite{Zevenhoven2018}.
 
 %Förbränning av biomassa är något som förknippas med CO$_2$ neutrala utsläpp och även i vissa fall negativa CO$_2$ utsläpp i de fall koldioxidlagring nyttjas som exempelvis en kemcyklisk förbränning\cite{Zevenhoven2018}.

%Vid analys av LD-slagg som nyttjats som bäddmaterial i Chalmers 12 MW cirkulerad fluidiserad bädd-förbrännare hittades ett skal av vanadin och fosfor kring partiklarnas yttre hölje med SEM-EDX-teknik. Det finns skäl att testa om detta skulle kunna underlätta lakning av båda dessa komponenter.

\subsection{Svenskt stål}

Stålindustrin har utvecklats rigoröst sedan de rudimentära stenugnarna som användes på medeltiden för att konvertera järnmalm till stål. Numera finns ett stort urval av processer som kan tillämpas baserat på vilken typ av malm som används, vilka förutsättningar som finns på plats och vilken slutprodukt som önskas \cite{Pehlke2014a}.

Stålprocessen börjar med att järnmalm grävs upp ur jordskorpan. Malmen, som består av olika former av järnoxid, anrikas i flera steg. I slutet av denna process fås järnpellets med homogen storlek och sammansättning som sedan kan användas vid tillverkning av råjärn i en masugn. Järnpelletsen innehåller ca 65\% järn \cite{RobertVikman}.

Masugnen är det första stora steget på vägen till färdigt stål. Processen är noggrannt styrd och kontrollerad, men är i grunden förbluffande lik de system som användes redan på medeltiden. Järnpellets, koks och slaggbildare skiktas i lager i masugnen som eldas nerifrån med en forcerad luftström som ger en mycket hög förbränningstemperatur. När masugnens olika lager sjunker neråt matas den uppifrån med nytt material. I botten av ugnen samlas det smälta järnet och kan tappas av. Det avtappade järnet kallas råjärn och innehåller ca 4\% kol. Rent kemiskt reduceras järnmalmen med koks som reduktionsmedel enligt reaktion \eqref{eq:järnoxid} nedan.

\begin{equation}
    C+FeO \rightarrow CO + Fe(l)
    \label{eq:järnoxid}
\end{equation}

Särskilt intressant för denna studie är syrgasprocessen eller LD-processen, Linz Donawitz-processen, som uppfanns i Österrike på tidigt 1950-tal \cite{Jernkontoret2000del1}.
Processens namn kommer från de två Österrikiska städerna som först tillämpade processen kommersiellt. Denna process är ett av reningsstegen som följer på masugnen för att producera smidbart stål som i slutändan har en kolhalt på 0,04-0,8\% \cite{Jernkontoret2000del2}.
Råjärnet behöver gå igenom en oxidationsprocess för att oxidera ut det oönskade kolet enligt reaktion \eqref{eq:kolox} nedan.

\begin{equation}
    C +\frac{1}{2} O_{2}(g) \rightarrow CO(g)
    \label{eq:kolox}
\end{equation}

I LD-processen blåses syrgas ner i järnsmältan och åstadkommer förutom värmetillförsel den önskade oxideringen enligt \eqref{eq:kolox}. En av fördelarna med denna metod är den effektiva omrörningen av smältan och slagget som finns däri.

Slagget som nämns ovan är vad man kallar de komponenter som inte kommer ingå i det färdiga stålet och som man önskar avskilja. Under LD-processen och många andra metoder spelar dock slagget en avgörande roll  \cite{Slagg-Jernkontoret}. 
Slagget som bildas flyter ovanpå smältans yta till följd av densitetsskillnader. På ytan fyller slagget två viktiga funktioner, det fungerar som en värmesköld för smältan och är samtidigt en barriär mot luften som inte ska komma åt och oxidera järnsmältan. Utöver den skyddande rollen kan slagg binda in olika föroreningar som inte är önskvärda i stålet. För att åstadkomma detta kan olika slaggbildare tillsättassom är anpassade för att binda in diverse föroreningar beroende på vilka råmaterial som används. Främst används kalksten för detta \cite{RobertVikman}.

\subsection{LD-slagg}
LD-slagg är det som bildas vid en så kallad LD-konvertering som är en del av framställningen av stål från järnmalm. Denna restprodukt innehåller fortfarande en hel del användbara ämnen se, Tabell \ref{tab:LDslagg},  som till exempel fosfor och vanadin. Svensk järnmalm innehåller sällsynt stora mängder vanadin vilket medför att även LD-slagget har kvar stora mängder av det. År 2015 medförde stålindustrin att 309 362 ton av LD-slagg bildades som står för 16\% av alla restprodukter från stålindustrin \cite{Jernkontoretrest}. Där den överväldigande majoriteten av LD-slaggen läggs på deponi.

\begin{table}[H]
\caption{Sammansättning, i vikt\% , av obehandlad LD-slagg som från Sverige och gentemot från Nordamerika \cite{Proctor2000}. }
\begin{tabular}{|l|l|l|l|l|l|l|l|l|l|l|}
\hline
Provbehandling                                                                 & Fe    & Ca    & Mg   & Mn   & Al   & V    & K    & P    & Si   & S    \\ \hline
\begin{tabular}[c]{@{}l@{}}Svenskt\\ LD-slagg \\ obhandlat\end{tabular}        & 17.07 & 31.73 & 5.88 & 2.64 & 0.76 & 1.51 & 0.04 & 0.25 & 5.61 & 0.1   \\ \hline
\begin{tabular}[c]{@{}l@{}}Nordamerikanskt\\ LD-Slagg\\ obhandlat\end{tabular} & 28.55 & 43.4  & 8.57 & 5.1  & 3.69 & 0.15 & N/A  & 0.49 & 9.24 & 0.17 \\ \hline
\end{tabular}
\label{tab:LDslagg}
\end{table}

\subsection{Förbränningstekniker}
%Nyckelord: fast material, fluidisering, effektivitet, sand, agglomeration, negative emissions

Effektivare förbränningsprocesser är nödvändigt för att dels tackla de klimatproblem som kan relateras till utsläpp av växthusgaser, men också de ekonomiska aspekter som kan relateras till drift av anläggningar.

Detta har öppnat upp för diverse förbränningstekniker där syrebärare spelat en central roll i förbättringsarbetet. Syrebäraren, som kan vara metalloxid, har i uppgift att minska lufttillförsel i processen vilket resulterar i mindre NO$_X$-utsläpp och billigare drift \cite{doi:10.1021/acs.energyfuels.7b00197}.

\subsubsection{Chemical-looping combustion}
Chemical-looping combustion, CLC, på svenska kemcyklisk förbränning, är en teknik vars syfte är att lagra CO$_2$ i en förbränningsprocess. Två stycken fluidiserade bäddreaktorer; en för bränsletillförsel och en annan för lufttillförsel, kan nyttjas för att separera CO$_2$ och H$_2$O i förbränningsprocessen.

För att transportera syre från luft- till bränslereaktorn nyttjas en syrebärare. Syrebäraren cirkuleras fram och tillbaka mellan reaktorerna för att växelvis oxideras och reduceras i processen. Förloppet illustreras i Figur \ref{fig:CLC} nedan.

\begin{figure}[H]
    \centering
    \includegraphics[scale=0.4]{CLC.png}
    \caption{Konceptuell skiss över en CLC-process}
    \label{fig:CLC}
\end{figure}

Oxidationen och reduktionen av syrebäraren kan beskrivas i ett reduktionssteg \eqref{eq:reduktion} följt av ett oxidationssteg \eqref{eq:oxidation} enligt reaktionschemat nedan

\begin{equation}
C_nH_2m+(2n+m)Me_xO_y \rightleftharpoons  CO_2+mH_2O+(2n+m)Me_xO_{y-1}
   \label{eq:reduktion}
\end{equation}
\begin{equation}
    O_2 + 2Me_xO_{y-1} \rightleftharpoons  2Me_xO_y
    \label{eq:oxidation}
\end{equation}

där CnH$_2$m är ett godtyckligt bränsle och Me är den metall som agerar syrebärare \cite{OresHenrik}.

\subsubsection{Oxygen Carrier Aided Combustion}
I konventionella fluidiserande bäddpannor så används sand som bäddmaterial. Tanken med Oxygen Carrier Aided Combustion (OCAC) är att man istället för sand använder ett syrebärande material, som ofta är en metalloxid \cite{doi:10.1021/acs.energyfuels.7b00197}. Detta gör att bäddmaterialet även fördelar syre jämnt i pannan och inte bara värme.

\subsubsection{Förbränning av biomassa}
Det som skiljer förbränning av biomassa från många andra bränslen är att biomassa har större variation på sin sammansättning \cite{Ashcomp}. Biomassa innehåller framförallt alkaliska ämnen som i en fluidiserad bäddpanna kan reagera med bäddmaterialet och då resultera i att bäddmaterialet agglomererar och då förhindrar fluidisering \cite{Alkali}-\cite{glas}. Detta innebär att pannan måste stannas för rengöring och utbyte av bäddmaterial måste ske.

Biomassaförbränning i CLC förknippas även med negativa nettoutsläpp av CO$_2$. Efter förbränningen i CLC:n förvaras CO$_2$ i hålrum under marken för att permanent tas bort ur atmosfärens kolkretslopp, vilket i slutändan leder till ett negativt nettoutsläpp av CO$_2$ \cite{Bui2018}.
%-CLC biomassa negativ emission
% Beor på att man förvarar CO2 i bergrund
%

\subsubsection{Syrebärande bäddmaterial}
%Här pratar vi om olika malmer
%Dit kommer vi till LD-slagg
Det faktum att bäddmaterial agglomererar vid biomassaförbränning resulterar i att bäddmaterialet behöver bytas ut med jämna mellanrum. Därför är billiga och effektiva syrebärare av intresse.  
Den skall inte bara uppvisa hög reaktivitet med avseende på oxidation och reduktion, utan även vara billig och ha en så låg miljöbelastning som möjligt. 

Malmer med höga halter övergångsmetaller som antingen mangan eller järn har visat sig fungera väl som syrebärare.
En järnbaserad mineral som i huvudsak består av FeTiO$_3$, ilmenit,  har varit av intresse för en omfattande mängd studier, och även andra ämnen innehållande järn och mangan är av intresse \cite{OresHenrik},\cite{Xu2017}.

%Deponerat LD-slagg ...för att möta behovet av en billig syrebärare.
Behovet av billiga syrebärare har lett till att LD-slagg setts som en möjlig kandidat på grund av dess järnhalt. Svenskt deponerat LD-slagg har därför testats i Chalmerspannan som syrebärande bäddmaterial för att möta behovet av en billig syrebärare \cite{chalmerspanna}.

%Ilmenite, hematite

\subsection{Lakning LD-slagg}

Opublicerad forskning av en syrebärarforskningsgrupp på Chalmers har visat att vanadin förflyttar sig mot ytan av LD-slagg partiklarna då de använts som syrebärare. Att koncentrationen på ytan ökar medför att det potentiellt skulle kunna underlätta utvinningen av ämnet. Det har även visat sig att fosfor från bränslet annsamlas på partikelytan, vilket medför att det potentiellt finns möjligheter att ta tillvara på fosfor i samband med vanadinutviningen. En möjlig separationsmetod skulle kunna vara lakning.

\subsubsection{Lakning}

Lakning bygger på att lösa upp en löslig del av ett material och på så vis skilja den från den olösliga delen. Det finns två huvudgrupper av lakning, där lösningsmedlet antingen rinner igenom lakgodset eller där lakgodset tillsätts i lösningsmedlet. Lösningsmedlet kommer antingen att lösa ämnena direkt eller reagera med dem så att produkten kan lösas i lösningsmedlet. Lakning gynnas av en ökad kontaktyta och en liten partikelstorlek är därmed gynnsamt \cite{lakning}. 




\subsubsection{Vanadin och fosfor}

Vanadin, grundämne nummer 23, är en övergångsmetall i det periodiska systemets femte grupp. Vanadin hittas i en handfull olika tillämpningsområden som exempelvis katalysator i en rad olika tillverkningsprocesser \cite{Pecoraro2014}.
Metallen har oxidationstillstånd, +2,+3,+4 och +5 \cite{Baroch2013}. Vilket har gjort att metallen varit av särskilt intresse i flödesbatterier och därför föremål för studier ända sedan förra århundradet \cite{Skyllas-Kazacos1987}.
Fördelen med de många oxidationstillstånden är att elektrolyterna i batteriets halvceller kan bestå av samma metall vilket är fördelaktigt med avseende på kontaminering av membran, elektrod och elektrolyter 
\cite{Lopez-Vizcaino2017}. Vanadin nyttjas dock mest som ett additiv i ståltillverkning vars syfte är att stärka stålet, avseende både hårdhet och hållfasthet \cite{Baroch2013}. Bland annat används en stor del av vanadinet som en legering för stål inom olika användningsområden, exempelvis balkar, armeringar och stålrör \cite{Pecoraro2014}. Vanligtvis så behöver vanadin tillsättas till stålet vid tillverkningen, men de höga mängderna naturligt förekommande vanadin i svensk järnmalm räcker för att producera högkvalitativt stål utan vidare tillsaster. 

Fosfor, grundämne nummer 15, är en icke-metall i periodiska systemets tredje grupp. Den främsta användningen för fosfor är inom jordbruket i form av konstgödsel. Utvinning av fosfor sker främst ur mineraler, då i form av fosfater. Eftersom fosfor är en begränsad resurs finns ett stort värde i att ta hand om alla materialströmmar som kan innehålla användbara kvantiteter av ämnet. LD-slagget har vid analys uppvisat en (specificera) halt fosfor som eventuellt skulle kunna tas till vara på i samband med lakningsprocessen avseende vanadin \cite{VanWazer2018}.

%I dagens samhälle ökar behovet av förnybar energi i form av exempelvis solkraft för att ersätta fossila alternativ. Detta yttrade sig tydligt i Sverige 2018 då intresset för solkraft ökade som ett svar på regeringens beslut om utökat investeringkostnadsstöd från 20 till 30 procent i solcellsinstallationer från och med den första januari 2018\cite{SverigesEnergimybn2009}.

%Problemet som kvarstår är dock att förnyelsebara energikällor i allmänhet och solceller i synnerhet inte producerar elektricitet i paritet med efterfrågan; förnyelsebara elnät är överlag känsliga mot de fluktuationer som uppstår i elnätet. Lösningen på ett av dagens energiproblem skulle alltså kunna lösas genom att lagra energin med flödesbatterier av vanadintyp\cite{Lopez-Vizcaino2017}.


