
\section{Syfte}
%Syftet med arbetet är således att testa om lakning av LD-slagg som befunnit sig i en fluidiserad bädd är en tillfredsställande metod för utvinning. Detta skulle kunna stänga materialkretslopp med avseende på inte bara vanadin utan eventuellt också fosfor och LD-slaggets livstid skulle kunna förlängas gentemot dagens spann. När lakning har utförts kommer det också analyseras i vilka typer av lakningsbara faser vanadinet finns i och hur dessa skulle kunna tänkas efterbehandlas. 

Detta arbetes syfte är att ta reda på i vilken grad det går att laka ur vanadin ur LD-slagg, som har används som bäddmaterial i Chalmerspannan, med hjälp av svavelsyra. Med detta kan det vara möjligt att öka LD-slaggets användningsområden och sluta materialkretsar som i nu läget deponeras på hög. 